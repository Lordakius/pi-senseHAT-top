%----------------------------------------------------------------------------------------
%	PACKAGES AND THEMES
%----------------------------------------------------------------------------------------

\documentclass{beamer}

\mode<presentation> {

% The Beamer class comes with a number of default slide themes
% which change the colors and layouts of slides. Below this is a list
% of all the themes, uncomment each in turn to see what they look like.

%\usetheme{default}
%\usetheme{AnnArbor}
%\usetheme{Antibes}
%\usetheme{Bergen}
%\usetheme{Berkeley}
%\usetheme{Berlin}
%\usetheme{Boadilla}
%\usetheme{CambridgeUS}
%\usetheme{Copenhagen}
%\usetheme{Darmstadt}
%\usetheme{Dresden}
%\usetheme{Frankfurt}
%\usetheme{Goettingen}
%\usetheme{Hannover}
%\usetheme{Ilmenau}
%\usetheme{JuanLesPins}
%\usetheme{Luebeck}
%\usetheme{Madrid}
%\usetheme{Malmoe}
%\usetheme{Marburg}
\usetheme{Montpellier}
%\usetheme{PaloAlto}
%\usetheme{Pittsburgh}
%\usetheme{Rochester}
%\usetheme{Singapore}
%\usetheme{Szeged}
%\usetheme{Warsaw}

% As well as themes, the Beamer class has a number of color themes
% for any slide theme. Uncomment each of these in turn to see how it
% changes the colors of your current slide theme.

%\usecolortheme{albatross}
%\usecolortheme{beaver}
%\usecolortheme{beetle}
%\usecolortheme{crane}
%\usecolortheme{dolphin}
%\usecolortheme{dove}
%\usecolortheme{fly}
%\usecolortheme{lily}
%\usecolortheme{orchid}
%\usecolortheme{rose}
%\usecolortheme{seagull}
%\usecolortheme{seahorse}
%\usecolortheme{whale}
%\usecolortheme{wolverine}

%\setbeamertemplate{footline} % To remove the footer line in all slides uncomment this line
\setbeamertemplate{footline}[page number] % To replace the footer line in all slides with a simple slide count uncomment this line

\setbeamertemplate{navigation symbols}{} % To remove the navigation symbols from the bottom of all slides uncomment this line
}

\usepackage{graphicx} % Allows including images
\usepackage{booktabs} % Allows the use of \toprule, \midrule and \bottomrule in tables
%\usepackage {tikz}
\usepackage{tkz-graph}
\usepackage{listings}
\GraphInit[vstyle = Shade]
\tikzset{
  LabelStyle/.style = { rectangle, rounded corners, draw,
                        minimum width = 2em, fill = yellow!50,
                        text = red, font = \bfseries },
  VertexStyle/.append style = { inner sep=5pt,
                                font = \normalsize\bfseries},
  EdgeStyle/.append style = {->, bend left} }
\usetikzlibrary {positioning}
%\usepackage {xcolor}
\definecolor {processblue}{cmyk}{0.96,0,0,0}
%----------------------------------------------------------------------------------------
%	TITLE PAGE
%----------------------------------------------------------------------------------------

\title[pi-Sensehat]{Implementing a top-interface for the Pi 3B} % The short title appears at the bottom of every slide, the full title is only on the title page

\author{Group 24} % Your name
\institute[Introduction to Embedded Systems 2019] % Your institution as it will appear on the bottom of every slide, may be shorthand to save space
{
Introduction to Embedded Systems\\ % Your institution for the title page
\medskip
}
\date{1 November 2019} % Date, can be changed to a custom date

\begin{document}

\begin{frame}
\titlepage % Print the title page as the first slide
\end{frame}

\begin{frame}
\frametitle{Overview} % Table of contents slide, comment this block out to remove it
\tableofcontents % Throughout your presentation, if you choose to use \section{} and \subsection{} commands, these will automatically be printed on this slide as an overview of your presentation
\end{frame}

%----------------------------------------------------------------------------------------
%	PRESENTATION SLIDES
%----------------------------------------------------------------------------------------

%------------------------------------------------

\section{Project Introduction}
\begin{frame}{Introduction}
    \begin{itemize}
				\item data extraction from the kernel interface
        \item 'top'-interface for the LED
        \item programs for load-simulation
    \end{itemize}
\end{frame}
\section{Overview}
\subsection{data extraction from the kernel interface}
\begin{frame}[fragile]{How can we acquire stats from the cpu?}
	\begin{itemize}
		\item pseudo filesystem proc as the kernel interface
		\item parsing files
	\end{itemize}

	\begin{lstlisting}
		fopen("/proc/stat");
		fgets(buffer);
		strtok(buffer);
	\end{lstlisting}
\end{frame}

\subsection{top-interface for the LED}
\begin{frame}{How can we display the information on the LED?}
There are multiple ways of displaying this information on the 8x8 grid. We
picked a simple approach.
  \begin{itemize}
		\item calculate the load (in \%) for every cpu
		\item set up threshold percentages for each of the 8 lights per row
		\item iterate through the cpu data and light the corresponding row
			accordingly
  \end{itemize}
\end{frame}

\begin{frame}[fragile]{How can we display the information on the LED?}
	\begin{lstlisting}[tabsize=2]
for(size_t i = 0; i <= MAX_CPU; i++) {
		perc = cpulist[i]->perc_use;

		if(perc > 0)
			set_led(i, 0, RGB565_GREEN);
		if(perc > 12.5)
			set_led(i, 1 , RGB565_GREEN);
		[...]
}
	\end{lstlisting}
\end{frame}
\section{programs for load-simulation}
\begin{frame}[fragile]{How can we check if our program reacts to changing workload?}
	We have written a few extra programs that utilize forks and some random
	calculations to change the workload. 

	This was intended to be controlled by the joystick, but discarded as the 
	change in the workload is no goal of the program itself.
	\begin{lstlisting}[tabsize=2]
	int main(void) {
		fork();
		size_t i = 0;
		while(1)
			i++;
	}
	\end{lstlisting}
\end{frame}
\section{future work}
\begin{frame}{There are still many improvements possible, including:}
\begin{itemize}
	\item display of cpu-data on a webserver
	\item additional problem: proper save of acquired data
	\item more datasets (e.g. memory load, swap)
\end{itemize}
\end{frame}
\begin{frame}{Interested? Check out the codebase}
	\url{https://github.com/Lordakius/pi-senseHAT-top}
\end{frame}

\end{document}
